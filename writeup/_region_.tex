\message{ !name(writeup.tex)}% Short Sectioned Assignment
% LaTeX Template
% Version 1.0 (5/5/12)
%
% This template has been downloaded from:
% http://www.LaTeXTemplates.com
%
% Original author:
% Frits Wenneker (http://www.howtotex.com)
%
% License:
% CC BY-NC-SA 3.0 (http://creativecommons.org/licenses/by-nc-sa/3.0/)
%
%%%%%%%%%%%%%%%%%%%%%%%%%%%%%%%%%%%%%%%%%

%----------------------------------------------------------------------------------------
%	PACKAGES AND OTHER DOCUMENT CONFIGURATIONS
%----------------------------------------------------------------------------------------

\documentclass[paper=a4, fontsize=11pt]{scrartcl} % A4 paper and 11pt font size

\usepackage[T1]{fontenc} % Use 8-bit encoding that has 256 glyphs
\usepackage[english]{babel} % English language/hyphenation
\usepackage{amsmath,amsfonts,amsthm} % Math packages


\usepackage{graphicx}
\usepackage{subfig}
\usepackage{amssymb}
\usepackage{hyperref}
\usepackage{float}

\usepackage{multicol}
\usepackage{mdwlist}
\usepackage{fancyhdr} % Custom headers and footers
\pagestyle{fancyplain} % Makes all pages in the document conform to the custom headers and footers
\fancyhead{} % No page header - if you want one, create it in the same way as the footers below
\fancyfoot[L]{} % Empty left footer
\fancyfoot[C]{} % Empty center footer
\fancyfoot[R]{\thepage} % Page numbering for right footer
\renewcommand{\headrulewidth}{0pt} % Remove header underlines
\renewcommand{\footrulewidth}{0pt} % Remove footer underlines
\setlength{\headheight}{6pt} % Customize the height of the header

\numberwithin{equation}{section} % Number equations within sections (i.e. 1.1, 1.2, 2.1, 2.2 instead of 1, 2, 3, 4)
\numberwithin{figure}{section} % Number figures within sections (i.e. 1.1, 1.2, 2.1, 2.2 instead of 1, 2, 3, 4)
\numberwithin{table}{section} % Number tables within sections (i.e. 1.1, 1.2, 2.1, 2.2 instead of 1, 2, 3, 4)

\setlength\parindent{0pt} % Removes all indentation from paragraphs - comment this line for an assignment with lots of text

%----------------------------------------------------------------------------------------
%	TITLE SECTION
%----------------------------------------------------------------------------------------

\newcommand{\horrule}[1]{\rule{\linewidth}{#1}} % Create horizontal rule command with 1 argument of height

\title{	
\normalfont \normalsize 
\textsc{UC Berkeley, Computer Science} \\ [25pt] % Your university, school and/or department name(s)
\horrule{0.5pt} \\[0.4cm] % Thin top horizontal rule
\huge Gender Classification of Handwritten Text \\ % The assignment title
\horrule{2pt} \\[0.5cm] % Thick bottom horizontal rule
}

\author{Peter Cheng, Jeff Tsui, Alice Wang} % Your name

\date{\normalsize\today} % Today's date or a custom date

\begin{document}

\message{ !name(writeup.tex) !offset(405) }
\section{Results}
\label{sec:results}
In this section, we apply 4 classification techniques to 4 different
feature sets, and compare their results. Because our features are
independently acquired on a per-line basis, and the end goal is to
classify each document, we decided upon three ways to aggregate our
features and determine their performance. The first is simply to
classify each line separately, and consider our accuracy to be the
percentage of lines accurately classified. The second method is to
average the feature vectors for all lines within a document, and end
with an ``average'' line feature vector for each document. The third
method is to classify each line separately, but classify each document
based on the majority vote of its line classifications. As a
benchmark, we also acquired Kaggle's 7067-length feature vectors for
each document. Each of these 4 feature sets was run through the
following classification techniques: linear SVM, SVM with an RBF
kernel, random forest, and k nearest neighbors. Accuracies of each can
be seen in Figure \ref{fig:resultsTable}.

\begin{figure}
\begin{center}
\begin{tabular} { | c | c | c | c | c | }
  \hline
     &   SVM-Linear   &   SVM-RBF   &   Random Forest   &   kNN \\ \hline
\end{tabular}
\end{center}
\caption{Accuracy of each classification method on each feature set}
\label{fig:resultsTable}
\end{figure}
\message{ !name(writeup.tex) !offset(438) }

\end{document}